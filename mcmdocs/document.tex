%%%%%%%%%%%%%%%%%%%%%%%%%%%%%%%%%%%%%%%%%%%%%%%%%%%%%%%%%
%                          导言区                        %
%%%%%%%%%%%%%%%%%%%%%%%%%%%%%%%%%%%%%%%%%%%%%%%%%%%%%%%%%

\documentclass[13pt]{ctexart} % 文档类
\usepackage{geometry} % 设置页面
\usepackage{graphicx} % 插图片
\graphicspath{{C:/Users/zheng/Desktop/mcm-2021/mcmdocs/figs/}}
\renewcommand{\figurename}{Figure} % 设置标题 重命名为英文
\renewcommand{\tablename}{Table}
\renewcommand{\contentsname}{Contents}
\usepackage{changepage} % 设置摘要页缩减
\usepackage{fontspec} % 便于修改字体
\usepackage{fancyhdr} % 设置页眉页脚
\usepackage{lastpage}
\pagestyle{fancy} % 清空页眉页脚
\usepackage[shortlabels]{enumitem} % 设置列表缩进
\usepackage{titlesec} % 设置修改默认的section标题大小
\titleformat*{\section}{\LARGE}
\titleformat*{\subsection}{\Large}
\titleformat*{\subsubsection}{\Large}
\usepackage{amsmath} % 使用数学宏包
\usepackage{array} % 设置表格的列格式
\usepackage{booktabs} % 三线表宏包
\usepackage{etoolbox} % 设置参考文献不输出默认名
\patchcmd{\thebibliography}{\section*{\refname}}{}{}{} % 引入网站作为参考文献
\usepackage{url} % 设置等宽的代码字体
%\setmonofont{IBM Plex Mono}
% 建议这个,但我系统没这个字体,且懒得折腾了 windows用户请
\setmonofont{Courier New}
% 颜色
\usepackage{xcolor} % 代码高亮方案宏包
\usepackage{listings}
\definecolor{CPPLight}  {HTML} {686868}
\definecolor{CPPSteel}  {HTML} {888888}
\definecolor{CPPDark}   {HTML} {262626}
\definecolor{CPPBlue}   {HTML} {4172A3}
\definecolor{CPPGreen}  {HTML} {487818}
\definecolor{CPPBrown}  {HTML} {A07040}
\definecolor{CPPRed}    {HTML} {AD4D3A}
\definecolor{CPPViolet} {HTML} {7040A0}
\definecolor{CPPGray}   {HTML} {B8B8B8}
\lstset{
	basicstyle=\ttfamily,
	breaklines=true,
	framextopmargin=50pt,
	frame=bottomline,
	columns=fixed,
    %numbers=left,                                       % 在左侧显示行号
	frame=none,                                          % 不显示背景边框
	backgroundcolor=\color[RGB]{255,255,255},            % 设定背景颜色
	keywordstyle=\color[RGB]{40,40,255},                 % 设定关键字颜色
	numberstyle=\footnotesize\color{darkgray},           % 设定行号格式
	commentstyle=\itshape\color[RGB]{0,96,96},                % 设置代码注释的格式
	stringstyle=\slshape\color[RGB]{128,0,0},   % 设置字符串格式
	showstringspaces=false,                              % 不显示字符串中的空格
	language=python,                                     % 设置语言
	morekeywords={alignas,continute,friend,register,true,alignof,decltype,goto,
		reinterpret_cast,try,asm,defult,if,return,typedef,auto,delete,inline,short,
		typeid,bool,do,int,signed,typename,break,double,long,sizeof,union,case,
		dynamic_cast,mutable,static,unsigned,catch,else,namespace,static_assert,using,
		char,enum,new,static_cast,virtual,char16_t,char32_t,explict,noexcept,struct,
		void,export,nullptr,switch,volatile,class,extern,operator,template,wchar_t,
		const,false,private,this,while,constexpr,float,protected,thread_local,
		const_cast,for,public,throw,std},
	emph={map,set,multimap,multiset,unordered_map,unordered_set,numpy,graph,path,append,extend,
		unordered_multiset,unordered_multimap,vector,string,list,deque,
		array,stack,forwared_list,iostream,memory,shared_ptr,unique_ptr,
		random,bitset,ostream,istream,cout,cin,endl,move,default_random_engine,
		uniform_int_distribution,iterator,algorithm,functional,bing,numeric,},
	emphstyle=\color{CPPViolet},
}

%%%%%%%%%%%%%%%%%%%%%%%%%%%%%%%%%%%%%%%%%%%%%%%%%%%%%%%%%
%                         正文区                         %
%%%%%%%%%%%%%%%%%%%%%%%%%%%%%%%%%%%%%%%%%%%%%%%%%%%%%%%%%
\begin{document}
%%%%%%%%%%%%%%%%%%%%%%%%%%扉页%%%%%%%%%%%%%%%%%%%%%%%%%%
\newgeometry{top = 1cm, right = 2.54cm, left = 2.54cm, bottom = 2.54cm}
% 第一页的字体为times new roman
\setmainfont{Times New Roman}
\thispagestyle{empty}
% 扉页抬头
\begin{table}[h]
    \quad { }  \begin{minipage}[t]{5.5cm}
        % arraystretch 是调节列高
        \begin{tabular}[t]{>{\centering\arraybackslash}b{10em}}
            \fontsize{12pt}{10pt}\selectfont \textbf{Problem Chosen} \\ [2pt]
            {\color{red} \fontsize{20pt}{10pt}\selectfont B} % 问题
        \end{tabular}
    \end{minipage}
    \begin{minipage}[t]{5.2cm}
        \begin{tabular}[t]{>{\centering\arraybackslash}p{10em}}
            \fontsize{12pt}{10pt}\selectfont \textbf{2021}    \\ [-2pt]
            \fontsize{12pt}{10pt}\selectfont \textbf{MCM/ICM} \\ [-2pt]
            \fontsize{12pt}{10pt}\selectfont \textbf{Summary Sheet}
        \end{tabular}
    \end{minipage}
    \begin{minipage}[t]{3cm}
        \begin{tabular}[t]{>{\centering\arraybackslash}b{12em}}
            \fontsize{12pt}{10pt}\selectfont \textbf{Team Control Number} \\ [2pt]
            {\color{red} \fontsize{21pt}{10pt}\selectfont 2120710} % 队伍控制号
        \end{tabular}
    \end{minipage}
\end{table}
\vspace{-20pt}
\noindent{\rule{\textwidth}{0.5mm}}
% 大标题
{\centering\fontsize{18}{16}\selectfont\textbf{{Fighting Wildfire with Unmanned Aerial Vehicle}}

    % summary
    \vspace{10pt}
    \fontsize{13}{10}\selectfont\textbf{{Summary}}\par}
\vspace{10pt}

% summray正文
\fontsize{13}{12.5}\selectfont %summary正文字体 13 pt
\begin{adjustwidth}{1cm}{1cm}
    \indent { }{ }{ }{ }{ }{ }
    在这里写summary

    % 关键词
    \vspace{15pt}
    \textbf{key words} : 关键词1; 关键词2; 关键词3
\end{adjustwidth}

%%%%%%%%%%%%%%%%%%%%%%%%%%MEMO页%%%%%%%%%%%%%%%%%%%%%%%%%%
\newpage
% 开始写 memo 信
% 更换字体为 palatino 也可以不换
\setmainfont{texgyrepagella-regular.otf}
\newgeometry{left = 3.5cm, right = 3.5cm}
\thispagestyle{empty}
% memo标题+信息
{\centering \fontsize{18pt}{14pt}\selectfont \textbf{Budget Request}\par}

\noindent FROM: Team {} 2120710 , MCM\\
\noindent To: The group of Governors\\
\noindent Date: February 8, 2020
% request正文
\vspace{10pt}
\\
这里是br正文。

% 信多出一页,清理页眉页脚
\thispagestyle{empty}
% 信的结尾
{\raggedleft
    Sincerely yours,\\
    MCM Team 2120710\par
}

%%%%%%%%%%%%%%%%%%%%%%%%%%目录页%%%%%%%%%%%%%%%%%%%%%%%%%%
\newpage
\thispagestyle{empty}
\tableofcontents
%%%%%%%%%%%%%%%%%%%%%%%%%%正文页%%%%%%%%%%%%%%%%%%%%%%%%%%
\newpage
% 目录页后面是第一页
\setcounter{page}{1}

% 开始写正文
% 设置正文的页边距
\newgeometry{top=3cm, left=3.5cm, right=3.5cm}
% 设置正文的页眉页脚
\fancyhf{}
\fancyhead[C]{ }
% 此处修改右上角页码
\fancyhead[R]{Page \thepage\ of \pageref{LastPage}}
\fancyhead[L]{Team \# 2120710}
\fancyfoot[C]{\bfseries\thepage}
%%%%%%%%%%%%%%%%%%%%%%%%%%%%%%%%%%%%%%%
%% Introduction
\textbf{\section{Introduction}}
% .1
\textbf{\subsection{Restatement of the Problem}}
Many people...Therefore we are facing the following problems:
\begin{itemize}
    \item aaaaaa
    \item bbbbbbb
\end{itemize}
% .2
\textbf{\subsection{Our Works}}
\begin{itemize}
    \item aaaaaa
    \item bbbbbbb
\end{itemize}
\fancyfoot[C]{\bfseries\thepage}
%%%%%%%%%%%%%%%%%%%%%%%%%%%%%%%%%%%%%%%
%% Assumptions and Notations
\textbf{\section{Assumptions and Notations}}
\vspace{-10pt}
\textbf{\subsection{Assumptions}}
Due to the lack of necessary data, we make the following assumptions to help us perform modeling:

\begin{itemize}[itemsep=0.3ex, leftmargin=1.2cm]
    \item[1.] The circumstance remain unchanged in the time interval we investigated.
    \item[2.] We omit the possibility of any other kinds of aerial vehcle or flying creature hitting our UAV.
    \item[3.] Accroding to Bureau of Meteorology of Australian Government, litghting is the major causation of bushfire in some area, Victoria included.[**] Based on this fact, we evaluat the possibility for a certain place to catch fire with the possibility of a lightning to occour there.
    \item[4.]  We adopt the Equal Possibility Hypothesis when our UVAs are patrolling for the purpose of monitoring any outbreak of fire.Under this hypothesis, an area of high possibility to catch fire indicates the frequency of fire outbreak here is high, thus the command center should pay closer attention to this area to alarm fire outbreaks timely.
    \item[5.] All UAVs are equipped with a timer.
    \item[6.] All UAVs are directed by a preprogrammed system given by us, which means they are all automatic.
    \item[7.] Staffes are always available in any charging stations, which guarantees the UVAs will always work in the stanterd situation.
    \item[8.] A drone can carry either a set of thermal imaging cameras and telemetry sensors or a radio repeater. The former combination can and can only detect any fire outbreak, while the latter can and can only extend the valid zone of radio wave signals.
\end{itemize}
\textbf{\subsection{Notations}}
Here are all the notations and their meanings in this paper.
\begin{table}[h]
    \centering
    \vspace{3pt}
    \begin{tabular}{>{\centering\arraybackslash}p{5em}>{\centering\arraybackslash}p{30em}}
        \toprule % 绘制第一条线
        Symbol                & Meaning                                                \\ \midrule
        $(x,y,z)$             & Coordinates                                            \\
        $M(i,x,y,z,t)$        & Coordinates matrix of $i$ -th SSA at time $t$          \\
        $P_{hexa}(x,y,z,t)$   & Coordinates matrix of front-line personnel at time $t$ \\
        $P(j,w,g,t)$          & 前线人员的经纬度坐标                                   \\
        $Posi(j,w,g)$         & 维多利亚省的经纬度坐标                                 \\
        $h(x,y,z)$            & 地点高度                                               \\
        $S(x,y,z)$            & 过去五年着火严重程度,约化火痕面积                     \\
        $F(x,y,z)$            & 森林覆盖程度,用单位密度植物密度计                     \\
        $Strength(i,x,y,z,t)$ & t时刻第i个中继器在(x,y)处的信号强度                 \\
        $E(x,y,z,N)$          & 有N个无人机在一点的覆盖效度                            \\
        $Slop(x,y,z)$         & 一点的最大坡度                                         \\
        $\beta(x,y,z)$        & 梯度权重                                               \\
        $\Gamma$              & 梯度因子                                               \\
        $Nmp(j)$              & 第j个区域的人-机信号衰减率                             \\
        $Nmm(i,k)$            & 第i个和第k个无人机之间的信号衰减频率                   \\

        \bottomrule
    \end{tabular}
\end{table}
%%%%%%%%%%%%%%%%%%%%%%%%%%%%%%%%%%%%%%%
%% Model Construction
\textbf{\section{Model Construction}}
blablablabla
%%%%%%%%%%%%%%%%%%%%%%%%%%%%%%%%%%%%%%%
%% Conclusion
\textbf{\section{Conclusion}}
We build a.....interesting findings:\cite{entroy}

\begin{itemize}
    \item aaaaaaa
    \item aaaaaaa
\end{itemize}

%%%%%%%%%%%%%%%%%%%%%%%%%%参考文献%%%%%%%%%%%%%%%%%%%%%%%%%%
% 因为不输出此部分到目录 \addcontentsline{}{}{}是添加此标题到目录
\newpage
\textbf{\section*{References}\addcontentsline{toc}{section}{References}}
\fancyhf{}
\fancyhead[R]{ }
\fancyhead[L]{ }
\bibliography{books}
\Large
\bibliographystyle{IEEEtran}
%%%%%%%%%%%%%%%%%%%%%%%%%%附录%%%%%%%%%%%%%%%%%%%%%%%%%%
\newpage
\textbf{\section*{Appendices}\addcontentsline{toc}{section}{Appendices for Code and Data}}
\fontsize{13pt}{12.5pt}\selectfont
Here is Code we used in our model, which python is the main development language.
\vspace{7pt}
\textbf{\subsection*{Appendices A}}
\begin {figure}[h]
\centering % 居中显示
\includegraphics[width=15cm,height=12cm]{8.png}
\caption{Transition matrix for synthetic opioid spread rate in West Virginia} % 标题
\end {figure}

\end{document}